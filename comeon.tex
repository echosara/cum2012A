\documentclass[UTF8,12.05pt]{ctexart}
\usepackage{graphicx}
\usepackage{geometry}
\geometry{a4paper,margin=2.8cm}
\usepackage{fancyhdr}
\usepackage{listings}
\usepackage{xcolor}
\usepackage{float}
\usepackage{amsmath}
\usepackage{fancyhdr}
\pagestyle{empty}
\begin{document}
\bibliographystyle{plain}
\begin{center}
  \zihao{3}\heiti葡萄酒评价问题统计学分析
\end{center}

\vspace{1cm}
\begin{abstract}
随着经济的发展和人民生活水平的提高,葡萄酒的需求在不断增长,葡萄酒的质量越来越多地影响着人们的健康。葡萄酒质量的评价是一项复杂的工作,它涉及到评酒员的感官评价和葡萄酒与酿酒葡萄的理化指标等方面,也是葡萄酒行业和应用数学所关注的一个重要问题。本文针对葡萄酒评价的问题,建立数学模型分析研究了四个问题。
\newline
针对问题一,问题要求比较两组评酒员的评价结果是否有差异,哪一组结果更可信,我们首先对数据进行预处理根据一组评酒员对葡萄酒的各个指标的评价得到葡萄酒的综合评价,然后用配对样本t检验比较两组评酒员对于葡萄酒的评价是否一致,每一组评价可信度的高低用克隆巴赫$\alpha$系数表示。
\newline
针对问题二,问题要求根据葡萄酒和酿酒葡萄的理化指标进行分级,我们对128个指标进行主成分分析,得到十三个主成分,然后根据主成分的得分用k-means算法对酿酒葡萄进行分级。
\newline
针对问题三,问题要求分析酿酒葡萄和葡萄酒的理化指标之间的联系,首先对不同的指标进行相关性分析,然后对关联紧密的变量进行多元线性回归。找到变量之间的联系,回归模型的准确度在77$\%$以上,所以可以用理化指标评价葡萄酒质量
\newline
针对问题四,
\begin{flushleft}
  \textbf{关键词:}主成分分析;\quad配对样本t检验;\quad k-means算法;\quad 克隆巴赫$\alpha$系数;相关性分析。
\end{flushleft}

\end{abstract}
\newpage
\pagestyle{plain}
\pagenumbering{arabic}
\setcounter{page}{1}
\section{\heiti\zihao{4}问题重述}
\normalsize确定葡萄酒质量时一般是通过聘请一批有资质的评酒员进行品评。每个评酒员在对葡萄酒进行品尝后对其分类指标打分,然后求和得到其总分,从而确定葡萄酒的质量。酿酒葡萄的好坏与所酿葡萄酒的质量有直接的关系,葡萄酒和酿酒葡萄检测的理化指标会在一定程度上反映葡萄酒和葡萄的质量。附件1给出了某一年份一些葡萄酒的评价结果,附件2和附件3分别给出了该年份这些葡萄酒的和酿酒葡萄的成分数据。请尝试建立数学模型讨论下列问题:
\begin{itemize}
  \item 问题1:分析附件1中两组评酒员的评价结果有无显著性差异,哪一组结果更可信?
  \item 问题2:根据酿酒葡萄的理化指标和葡萄酒的质量对这些酿酒葡萄进行分级。
  \item 问题3:分析酿酒葡萄与葡萄酒的理化指标之间的联系。
  \item 问题4:分析酿酒葡萄和葡萄酒的理化指标对葡萄酒质量的影响,并论证能否用葡萄和葡萄酒的理化指标来评价葡萄酒的质量。
\end{itemize}
\section{\heiti\zihao{4}问题假设}
\begin{enumerate}
  \item 每一位评酒员在评价时不会相互影响,且他们的评价结果地位均等;
  \item 葡萄酒的成分是近似均一不变的;
  \item 葡萄酒的编号和酿酒葡萄的编号是一一对应的。
  \item 葡萄酒评价各个指标满分分值的不同反映了不同指标的重要性
\end{enumerate}
\section{\heiti\zihao{4}符号说明}
\begin{table}[H]
  \centering
  \begin{tabular}{|c| c| c|}
  \hline
  % after \\: \hline or \cline{col1-col2} \cline{col3-col4} ...
  序号&关键符号&符号说明 \\ \hline
  1&$\large R$&葡萄酒酒样品综合评价指数 \\
  2&$\large \overline{R}$&葡萄酒酒样品综合评价指数平均值 \\
  3&$\large r_{i}$&葡萄酒第i项指标评价指数 \\
  4&$\large \sigma_{R}$&葡萄酒综合评价指数的样本方差 \\
  5&$\large \gamma $&样本之间的相关系数 \\
  6&$\large F_{i}$&主成分因子\\
  7&$\large \alpha$&克隆巴赫系数\\ \hline
\end{tabular}
\caption{符号说明}
\end{table}

\section{\heiti\zihao{4}模型的建立和求解}
\subsection{\heiti\zihao{-4}问题一的分析和解答}
\subsubsection{\heiti\zihao{-4}问题的分析}
问题一要分析两组评价结果的显著性差异,并比较结果的可信度。我们用配对样本t检验来描述两个样本平均值与其代表的总体的差异是否显著。用$\alpha$来描述可信度,比较两个评价谁的可信度比较高。
\subsubsection{\heiti\zihao{-4}模型的建立和求解}
\paragraph{数据预处理}首先对附件1中的数据进行预处理,采用算术平均法计算同一组评酒员对葡萄酒一个指标的评价得分,然后将每一个指标得分相加得到一组评酒员对这个葡萄酒样本的综合评价。葡萄酒的综合评价指数:
$$R=\sum_{i=1}^{i=n}r_{i}$$
其中R为葡萄酒的综合评价指数,$r_{i}$为葡萄酒每一个指标的得分。
\paragraph{t检验}
t检验是用t分布理论来推论差异发生的概率,从而比较两个平均数的差异是否显著。
分别对红葡萄酒和白葡萄酒做两组综合评价的配对样本t检验。
$$t=\frac{\overline{R_{1}}-\overline{R_{2}}}{\sqrt{\frac{\sigma_{R_{1}}^{2}+\sigma_{R_{2}}^{2}-2\gamma\sigma_{R_{1}}\sigma_{R_{1}}}{n-1}}}$$
其中$\overline{R}$表示酒样综合评价的平均值,$\sigma^{2}$表示样本方差,$\gamma$表示样本相关系数。
在显著性水平$\alpha=0.05$下判断两组评酒员的评价结果有无显著性差异,在$\mu_{1},\mu_{2}$未知的条件下检验假设:$H_{0}:\mu_{1}=\mu_{2},H_{1}:\mu_{1}\neq\mu_{1}$
得到如下结果。
\begin{figure}[H]
  \centering
  \includegraphics[width=6.00in]{tjianyan.png}
\end{figure}
从配对样本相关性的表格中可以看出,两组评酒员对红葡萄酒的评价有较强的相关性,相关系数为0.439,说明两组对红葡萄酒品质的好坏存在比较一致的看法。而对白葡萄酒的评价相关性较低,评价一致性较低。
\newline
对于红葡萄酒$t=2.039>t_{0.05}(27)=1.7033$,白葡萄酒$t=-2.076<-t_{0.05}(28)=-1.7011$都应拒绝$H_ {0}:\mu_{1}=\mu_{2}$,因此认为两组评酒员在评价结果上有显著差异。
\newline
\paragraph{可信度分析}
一般研究用$\alpha$系数反映一组内评价结果是否稳定一致。
$$\alpha=(\frac{k}{k-1})(1-\frac{\sum_{i=1}^{k}\sigma_{i}^{2}}{\sigma_{t}^{2}})$$
k:量表的项目数;
$\sigma_{i}^{2}$:第i个指标评价结果的组内离均差;
$\sigma_{t}^{2}$:所有结果的离均差。$\alpha$的不同的值表示不同的信度水平
\newline
$0.9<\alpha$信度佳
\newline
$0.7<\alpha<0.9$信度可
\newline
$0.5<\alpha<0.7$信度低
\newline
$\alpha<0.5$没有信度意义
\begin{figure}[H]
\begin{minipage}{0.5\linewidth}\centering
   \includegraphics[width=2in]{alpha1.png}\caption{第一组评价可信度}
\end{minipage}
\begin{minipage}{0.5\linewidth}\centering
  \includegraphics[width=2.4in]{alpha2.png}\caption{第二组评价可信度}
\end{minipage}
\end{figure}
第一组的克隆巴赫$\alpha$系数为0.858,第一组的克隆巴赫$\alpha$系数为0.770,两组评价的$\alpha$系数均在0.7以上,说明信度较好。且第一组评价要比第二组更可信。
\subsection{\heiti\zihao{-4}问题二的分析和解答}
\subsubsection{\heiti\zihao{-4}问题的分析}
问题二需要通过酿酒葡萄的理化指标和葡萄酒的质量对这些酿酒葡萄进行分级。我们需要综合考虑酿酒葡萄和葡萄酒的各个指标,寻找合理建立评价指标的标准。这里不同的指标数据有不同的量纲,所以首先对数据进行标准化处理,又因为这里的评价指标较多并且之间可能存在多重共线性关系,这会给数据处理带来不便。所以还要对数据进行降维,这里用的方法是主成分分析,最后根据用主成分分析得到的变量用K-means聚类算法对不同的酿酒葡萄进行分级。为了更好地反映每一种葡萄的品质,我们根据综合评价标准将酿酒葡萄分为4个等级(A为最优,D为最差),使得指标更加具体。
\subsubsection{\heiti\zihao{-4}模型的建立和求解}

\paragraph{主成分分析}
主成分分析(PCA)是将多项指标重新组合成一组新
的互相无关的几个综合指标, 据实际需要从中选取尽
可能少的综合指标,以达到尽可能多的反映原指标信息的分析方法。
对128个酿酒葡萄的理化指标和葡萄酒的品质指标进行主成分分析后,得到十三个主成分,累计贡献率为$70.357\%$,具有统计学上的意义。
\subparagraph{第一主成分}乙醛,乙酸甲酯,乙酸乙酯,3-甲基丁醛,乙醇,2-戊酮,乙酸-2-甲基丙基酯,甲苯,乙酸丁酯,2-甲基-1-丁醇-乙酸酯,乙酸戊酯,3-甲基-1-丁醇,1-戊酯,乙酸己酯,2-辛酮,4-己烯-1-醇-乙酸酯,1-己酯,乙酸庚酯,(Z)-3-己烯-1-醇,辛酸乙酯,6-甲基-5-庚烯-2-醇,乙酸辛酯,2-乙基-1-己醇,壬酸乙酯,5-甲基糠醛,萘,甲氧基苯基丙酮肟,1-甲氧基-4-(1-丙烯基)苯,苯乙醇。
 \subparagraph{ 第二主成分}蛋白质,花色苷,总酚,单宁,葡萄总黄酮,反式白藜芦醇,黄酮醇,杨梅黄酮,槲皮素,异鼠李素,果皮颜色L,果皮颜色b,果皮颜色C1,果皮颜色C2,果皮颜色C3,香气纯正度,香气浓度,香气质量,口感纯正度,口感浓度,口感质量,平衡。
 \subparagraph{ 第三主成分}氨基酸总量,天门冬氨酸,苏氨酸,丝氨酸,谷氨酸,甘氨酸,丙氨酸,缬氨酸,异亮氨酸,赖氨酸,组氨酸。
 \subparagraph{ 第四主成分}总糖,还原糖,果糖,葡萄糖,可溶性固形物,干物质含量。
 \subparagraph{ 第五主成分}(Z)-2-庚烯醛,(E)-2-己烯-1-醇,1-庚醇,反式-2-壬烯酸。
 \subparagraph{ 第六主成分}异亮氨酸,三氯甲烷,(E,Z)-2,6-壬二烯醛,(E)-2-壬烯-1-醇。

 \subparagraph{ 第七主成分}乙醛。
 \subparagraph{ 第八主成分}(Z)-3,7-二甲基-2,6-辛二烯醛,(R)-3,7-二甲基-6-辛烯醇, (E)-3,7-二甲基-2,6-辛二烯-1-醇。
 \subparagraph{ 第九主成分}果梗比,乙醛,己酸乙酯。
 \subparagraph{ 第十主成分}可滴定酸,固酸比。
  \subparagraph{第十一主成分}1-庚醇。
  \subparagraph{第十二主成分}脯氨酸,反式-2-壬烯酸。
  \subparagraph{第十三主成分}反式-2-壬烯酸。

进一步分析主成分的包含成分可知。第一主成分主要是挥发性芳香类物质;第二主成分主要描述葡萄酒质量和葡萄颜色;第三主成分主要是氨基酸,;第四主成分主要是糖类物质等。上述结果说明酿酒葡萄的理化指标和葡萄酒的质量中最具代表性的指标依次是芳香物质,葡萄酒质量,氨基酸和糖类等。

\paragraph{k-means聚类分析}
聚类分析(Cluster Analysis)是研究分类问题的多元统计方法
之一, 就是根据研究对象的特征把性质相近的个体归为一类, 使得同一类中的个体具有高度的同质性,不同类之间的个体具有高度的异质性的多元分析技术的总称。
其中常用k-means聚类算法,K-means算法是将样本聚类成k个cluster,具体算法描述如下:
1、随机选取k个聚类质心点(cluster centroids)
2、重复下面过程直到收敛
      对于每一个样例i,计算其应该属于的类
      $$c^{i}:=argmin||x^{(i)}-\mu_{j}||^{2}$$.
      对于每一类j,重新计算该类的质心
      $$\mu_{j}:=\frac{\sum_{i=1}^{m}1{c^{(i)}=j}x^{(i)}}{\sum_{i=1}^{m}1{c^{(i)}=j}}$$
以主成分分析所得到主成分作为变量进行k-means聚类分析,得到结果如下表所示。

 \begin{table}[H]
 \centering
 \caption{葡萄酒分级}
  \begin{tabular}{|c|c|c|c|}
    \hline
    % after \\: \hline or \cline{col1-col2} \cline{col3-col4} ...
    红葡萄酒葡萄样本 & 等级 & 白葡萄酒样本&等级\\
    \hline
    1&B&1&C \\ \hline
    2&C&2&B \\ \hline
    3&C&3&B\\ \hline
    4&C&4&D\\ \hline
    5& A&5&D\\ \hline
    6& C&6&C\\ \hline
    7& C&7&D\\ \hline
    8& C&8&C\\ \hline
    9& C&9&B\\ \hline
    10&C&10&C\\ \hline
    11&C&11& B\\ \hline
    12& B&12&A\\ \hline
    13&A&13&C\\ \hline
    14& A&14&A\\ \hline
    15& A&15&D\\ \hline
    16& C&16&B\\ \hline
    17& C&17&C\\ \hline
    18& B&18&D\\ \hline
    19& A&19&B\\ \hline
    20&C&20&C\\ \hline
    21& A&21&C\\ \hline
    22& A&22&C\\ \hline
    23&C&23&C\\ \hline
    24&C&24&B \\ \hline
    25& A&25&B\\ \hline
    26& A&26&B\\ \hline
    27& D&27&D\\ \hline
    & &28&B\\
    \hline
    \end{tabular}
    \end{table}
\subsection{\heiti\zihao{-4}问题三的分析和解答}
\subsubsection{\heiti\zihao{-4}问题分析}
问题三需要分析酿酒葡萄与葡萄酒理化指标之间的联系。首先用相关性分析了解变量之间(酿酒葡萄和葡萄酒的理化指标)关系密切程度。可以将葡萄酒的理化指标作为因变量,由相关分析得到的关系紧密的酿酒葡萄的理化指标作为自变量,用多元线性回归方程刻画这种关系。
\subsubsection{\heiti\zihao{-4}模型的建立和求解}
\paragraph{数据相关性分析}
采用分析软件SPSS24.0的Correlate中Bivariate分析,它不需要区分自变量和因变量,两个变量或者多个变量之间是平等的关系。相关性表格见附录1
\paragraph{多元线性回归分析}
多元线性回归是研究多个自变量与一个因变量间是否存在线性关系(相互依存关系),并用多元线性回归方程来表达这种关系 (或用回归方程定量地刻画一个因变量与多个自变量间的线性依存关系)的数学分析方法。
随机变量y与一般变量$x_{1},x_{2}...x_{3}$的线性回归模型为
$$y=\beta_{0}+\beta_{1}x_{1}+...+\beta_{p}x_{p}+\varepsilon$$
其中,$\beta_{0},\beta_{1},...,\beta_{p}$是p+1个未知参数,$\beta_{0}$称为回归常数,$\beta_{1},\beta_{2},...,\beta_{p}$称为回归参数。
多元线性回归方程的参数估计采用最小二乘估计,寻找参数$\beta_{0},\beta_{1},...
,\beta_{p}$的估计值,使得离差平方和
$$Q(\beta_{0},\beta_{1},...,\beta_{p})=\sum_{i=1}^{n}(y_[i]-\beta_{0}-\beta_{1}x_{i1}-...-\beta_{p}x_{ip})^{2}$$
达到极小,即$\hat{\beta}_{0},\hat{\beta}_{1},\hat{\beta}_{2},...,
\hat{\beta}_{p},$满足
$$Q(\hat{\beta}_{0},\hat{\beta}_{1},...,\hat{\beta}_{p})=\sum_{i=1}^{n}(y_[i]-\hat{\beta}_{0}-\hat{\beta}_{1}x_{i1}-...-\hat{\beta}_{p}x_{ip})^{2}$$
  $$ =minQ(\beta_{0},\beta_{1},...,\beta_{p})$$
求导后可得到回归参数的最小二乘估计为
$$\hat{\beta}=(X'X)^{-1}X'y$$
建立模型后需对参数进行检验,并不是所有变量对因变量有显著的影响,需要挑选出对因变量有显著性影响的自变量。其中逐步回归是最为常用的。逐步回归的基本思想是有进有出的。具体做法是将变量一个一个引入,当每引入一个自变量后,对已选入的变量要进行逐个检验,当原引入的变量由于后面变量的引入而变得不再显著时,要将其剔除。引入一个变量或从回归方程中剔除一个变量,为逐步回归的一步。每一步都要进行F检验,以确保每次引入新的变量之前回归方程中只包含显著的变量,这个过程反复进行,知道既无显著的自变量进入回归方程,也无不显著的自变量从回归方程中剔除为止。这样就避免了前进法和后退法各自的缺陷,保证了最后所得的回归子集是最优回归子集。

,最终得到的方程模型如下:

葡萄酒花色苷=1.529+2.439$\times$葡萄花色苷

葡萄酒单宁=0.508+0.134$\times$葡萄总酚+0.233$\times$葡萄单宁+0.240$\times$葡萄异鼠李素

葡萄酒总酚=0.952-0.015$\times$葡萄花色苷+0.236$\times$葡萄单宁

葡萄酒总黄酮=-1.414+0.420$\times$葡萄总酚

葡萄酒白藜芦醇苷=1.286+6.907$\times$葡萄顺式白藜芦醇

葡萄酒反式白藜芦醇苷=0.110+3.135$\times$葡萄反式白藜芦醇+4.836$\times$葡萄顺式白藜芦醇

葡萄酒顺式白藜芦醇苷=0.430+1.081$\times$葡萄白藜芦醇-1.201$\times$葡萄反式白藜芦醇苷-1.082$\times$葡萄顺式白藜芦醇苷

葡萄酒反式白藜芦醇=-0.158+0.001$\times$葡萄果穗质量

葡萄酒顺式白藜芦醇=-0.076+0.002$\times$葡萄花色苷

葡萄酒DPPH半抑制体积=-0.031+0.014$\times$总酚+0.008$\times$异鼠李素
\subsection{\heiti\zihao{-4}问题四的分析和解答}
\subsubsection{\heiti\zihao{-4}问题分析}
问题四要求分析酿酒葡萄和葡萄酒的理化指标对葡萄酒质量的影响,并论证能否用葡萄和葡萄酒的理化指标来评价葡萄酒的质量。和问题三一样,首先用相关分析了解葡萄酒品质与各个理化指标之间的关系紧密度,然后用葡萄酒感官评价作为因变量,理化指标作为自变量,用多元线性回归刻画关系\cite{qiantanputaojiuniangjiu}。
\subsection{\heiti\zihao{-4}模型的建立和求解}
\paragraph{数据相关性分析}
同样采用SPSS24.0对葡萄酒品质和葡萄和葡萄酒理化指标进行相关性分析,得到的结果见附录2
\paragraph{多元线性回归分析}
用SPSS进行变量进行多元线性回归分析后,得到的方程模型如下:

澄清度=3.427+0.007$\times$H1        0.991

色度=6.677-0.060$\times$总黄酮+0.008$\times$(E)-2-己烯-1-醇

香气纯正度=5.407-0.05$\times$单宁+0.214$\times$总酚-0.065$\times$总黄酮-0.013$\times$白藜芦醇+0.182$\times$顺式白藜芦醇-1.720$\times$DPPH半抑制体积-0.011$\times$3-甲基-1-丁醇-乙酸酯            0.793

香气浓度=3.578-0133$\times$单宁+0.234$\times$总酚-0.090$\times$总黄酮-0.160$\times$顺式白藜芦醇苷+0.014$\times$DPPH半抑制体积-0.145$\times$色泽a-0.067$\times$色泽b+0.165$\times$色泽c-0.005$\times$2-甲基-1-丙醇-0.009$\times$3-甲基-1-丁醇-乙酸酯+0.008$\times$柠檬烯       0。77

香气质量=12,875-0.077$\times$单宁+0.175$\times$总酚-0.1$\times$总黄酮-1.011$\times$DPPH半抑制体积+-0.011$\times$3-甲基-1-丁醇-乙酸酯+0.048$\times$十二酸乙酯  0.839

口感纯正度=3.848-0.018$\times$单宁+0.118$\times$总酚-0.055$\times$总黄酮+0.072$\times$白藜芦醇-0.212$\times$顺式白藜芦醇苷-1.859$\times$DPPH半抑制体积        0.903

口感浓度=6.292-0.077$\times$单宁+0.155$\times$总酚-0.033$\times$总黄酮-0.093$\times$顺式白藜芦醇苷-1.221$\times$DPPH半抑制体积-0.118$\times$色泽a-0.064 $\times$色泽b+0.129$\times$色泽c  0.903

口感持久度=6.081-0.007$\times$2-甲基-1-丙醇-0.009$\times$3-甲基-1-丁醇-乙酸酯+0.066$\times$正十三烷+0.004$\times$辛酸乙酯-0.004$\times$己酸乙酯  准确度95.3$\%$

口感质量=16.544-0.113$\times$单宁+0.140$\times$总酚-0.163$\times$总黄酮+0.158$\times$
白藜芦醇-0.595$\times$顺式白藜芦醇苷-1.611$\times$DPPH半抑制体积

平衡=8.561-0.054$\times$单宁+0.175$\times$总酚-0.044总黄酮-0.007$\times$白藜芦醇-1.219$\times$DPPH半抑制体积+0.295$\times$反式白藜芦醇+0.006$\times$色泽L+0.003$\times$色泽a-0.005$\times$色泽c

总体感官评价=69.313+0.789$\times$单宁+0.570$\times$总酚-0.885$\times$总黄酮+0.085$\times$白藜芦醇-13.578$\times$DPPH半抑制体积+0.046$\times$色泽L-0.342$\times$色泽a+0.274$\times$色泽c+0.143$\times$庚酸乙酯

\begin{table}[H]
  \centering
  \begin{tabular}{|c|c|}
  \hline
  % after \\: \hline or \cline{col1-col2} \cline{col3-col4} ...
  评价 & 预测准确度 \\ \hline
  澄清度 & 0.991 \\ \hline
  色度 & 0.981 \\ \hline
  香气纯正度 & 0.991 \\ \hline
  香气浓度 & 0.770 \\ \hline
  香气质量 & 0.839 \\ \hline
  口感纯正度 & 0.903\\ \hline
  口感浓度 & 0.903 \\ \hline
  口感持久度 & 0.953 \\ \hline
  口感质量 & 0.933 \\ \hline
  平衡 & 0.892 \\ \hline
  总体评价 & 0.913 \\
  \hline
\end{tabular}
  \caption{预测准确度}
\end{table}

每一个指标的准确度都在70$\%$以上说明回归模型各系数和常数具有统计学意义。理化指标能够去评价葡萄酒质量。
\section{\heiti\zihao{4}模型的优点与不足}
葡萄酒是由新鲜葡萄或葡萄汁经过酒精发酵而得到的一种含酒精饮料。葡萄酒质量是其外观、香气、口味、典型性的综合表现。一方面,酒中的糖、酸、矿物质和酚类化合物, 都具有各自独特的风味, 它们组成了葡萄酒的酒体;另一方面,酒中大量的挥发性物质,包括醇、酯、醛等,都具有不同浓度、不同
愉悦程度的香气, 葡萄酒最终的质量则是葡萄酒中各种成分协调平衡的结果。葡萄酒的成分之间存在着复杂的关系, 它们又与感官质量之间有着密切的联系。
首先对两组评价数据进行简化处理,然后对综合评分进行配对样本t检验,运用SPSS软件得到,两组评酒员对红葡萄酒和白葡萄酒的评价均不一致。针对数据的可信度分析,我们对两组数据进行克隆巴赫$\alpha$系数的分析,第一组数据的$\alpha$系数大于第二组数据的$\alpha$系数,所以第一组结果更可信。
在对葡萄酒分级时,首先我们对葡萄酒的各个指标进行主成分分析,将128个指标化成13个主成分,然后针对于主成分的得分进行k-means聚类分析。得到A,B,C,D四个等级。
在分析酿酒葡萄和葡萄酒的理化指标对葡萄酒质量的影响,用葡萄和葡萄酒的理化指标来评价葡萄酒的质量,首先对不同的指标之间进行相关性分析,然后进行多元线性回归,得到的预测模型的准确度均在70$\%$以上,所以可以用理化指标去描述质量。


\bibliography{bibfile}

\end{document} 